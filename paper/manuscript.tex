\documentclass[12pt,titlepage]{article}

%% THE USEPACKAGES NECESSARY FOR THIS EXAMPLE
%% NOTE THAT genetics_manu_style MUST BE CALLED AFTER mychicago
\usepackage{graphicx}
\usepackage{endfloat}
\usepackage{amsfonts}
\usepackage{mychicago}
\usepackage{subfigure}
\usepackage{genetics_manu_style}
\usepackage{authblk} % improved citations
\usepackage{setspace} %double spacing
\usepackage{lineno} % line numbers
\usepackage{booktabs} % book-quality tables
\usepackage{multicol} % list corresponding authors in two separate columns
\usepackage{hyperref} % web links

%% Probably unnecessary packages.
\usepackage[T1]{fontenc}
\usepackage{lmodern}
\usepackage{amssymb,amsmath}
\usepackage{ifxetex,ifluatex}
\usepackage{fixltx2e} % provides \textsubscript




%% THE MANUSCRIPT TITLE
\title{Omics-based Single Step Trait Prediction}


% Allow for multiple authors to share the same institution.
\newcommand*\samethanks[1][\value{footnote}]{\footnotemark[#1]}

\author{
  Matthias Westhues\thanks{Institute of Plant Breeding, Seed Science and Population Genetics, University of Hohenheim, 70599 Stuttgart, Germany},
  Claas Heuer\thanks{Institute of Animal Breeding and Husbandry, Christian-Albrechts-University Kiel, 24098 Kiel, Germany},
  Georg Thaller\samethanks[2],
  Rohan Fernando\thanks{Department of Animal Science, Iowa State University, 50011 Ames, Iowa, U.S.A.},
  Albrecht E. Melchinger\samethanks[1]
}


\renewcommand{\CorrespondingAddress}{%
    University of Hohenheim \\
    Institute of Plant Breeding, Seed Science and Population Genetics \\ 
    Fruwirthstr. 21 \\ 
    Stuttgart 70599, GERMANY \\
    Tel.: (+49) 0711 459-22334 \\
    Fax: (+49) 0711 459-22343 \\ 
    \texttt{melchinger@uni-hohenheim.de} \vfill}

\renewcommand{\RunningHead}{Single Step Prediction}
\renewcommand{\CorrespondingAuthor}{Prof. Dr. A. E. Melchinger}
\renewcommand{\KeyWords}{single step, omics, prediction, maize}




%% SOME COMMANDS FOR THE CONTENT OF THIS FILE.  NOT NECESSARY FOR
%% GENETIC_MANU_STYLE
\newcommand{\bp}{\mathbf{p}}
\newcommand{\LLL}{\mathcal{L}}




%% BEGIN DOC
\begin{document}


\maketitle
\doublespacing
\linenumbers



\begin{abstract}

  $\dots$

\end{abstract}



% Please add here a significance statement to explain the relevance of your work
\section{Article Summary}
$\dots$



% Introduction
\section{Introduction} 
% Rationale for predictions.
Genomic prediction, pioneered by \citeNP{Meuwissen2001}, has revolutionized
animal and plant breeding by a more efficient selection of promising candidates 
without phenotypes \cite{DeLosCampos2013,Garcia-Ruiz2016} and recently emerged
as an important tool in human diagnostics \cite{DeLosCampos2010,Vazquez2016}.
Its use is largely motivated by the high costs and oftentimes long duration of 
phenotyping, which can be reduced when covering all genotypes with a predictor 
while limiting phenotyping to a few genotypes with high precision
\cite{Kadam2016}.
The performance of all other genotypes is then forecast using model parameters
obtained from thorough resampling routines applied to the subset of individuals 
with full information \cite{Hastie2009}.
In animal breeding, genomic selection has increased the annual selection gain 
by $\approx$ 50-100\% for yield traits and by 300-400\% for traits with low 
heritability \cite{Garcia-Ruiz2016}.

% Incomplete predictors.
Breeding programs --- particularly in animal breeding --- are characterized by
the presence of one predictor type, which is available for all recorded
genotypes while other predictors are incomplete in that they cover only a 
subset of all genotypes \cite{Fragomeni2015}.
Usually, the complete predictor is pedigree information while genomic data are
the most prevalent incomplete predictor.
To utilize all available phenotypic information for model training, methods were 
developed that allowed for a combined use of genomic and pedigree information 
\cite{Hayes2009a,VanRaden2009}.

% Two-step prediction.
Initially, a two-step procedure was the most intuitive choice for accomplishing
this goal \cite{VanRaden2009}.
The first step consists of conventionally estimating breeding values based on
pedigree relationships and the phenotypes of relatives.
Individuals with accurate breeding values out of the first step (\textit{i.e.}
high number of offspring), enter the second step in which conventional breeding
values are regressed onto SNP genotypes.
The estimated model coefficients can then be used to predict genomic breeding 
values for solely genotyped and potentially very young individuals.
However, as the breeding values from step one enter the genomic prediction
model as phenotypes, one has to be careful about the residual error
distribution of such a model.
For instance, the range of the prediction errors will no longer be identically
distributed, \textit{i.e.}, they should be weighted differently or estimated 
individually \cite{Aguilar2010}.
One popular approach to account for this problem is deregressing the breeding
values and weighting the residuals according to the prediction error variances
\cite{Garrick2009}.

% Workings and advantages of the single step procedure.
Independently from each other, \citeNP{Legarra2009} and
\citeNP{Christensen2010} developed a single step BLUP method that blends the 
numerator relationship matrix $\mathbf{A}$ with the genomic relationship matrix
$\mathbf{G}$ in a mutual matrix $\mathbf{H}$ to use all predictor information
simultaneously.
Issues with this H-BLUP approach were the compatability of $\mathbf{A}$ with
$\mathbf{G}$ \cite{Christensen2012} and their weighting, which is not
trait-independent \cite{Vitezica2011,Ashraf2016}.
\citeNP{Fernando2014} derived the equivalent single step marker-effect-model to 
the single step breeding value model of \citeNP{Legarra2009} and 
\citeNP{Christensen2010}, which does not require weights for $\mathbf{A}$ and 
$\mathbf{G}$.
An additional advantage of this alternative formulation is the ability to 
implicitly model the structure of the imputation error.


% -   Mention scarcity of data on single step prediction in plant populations.
In animal breeding, single step prediction, particularly H-BLUP, has already 
become a routine procedure \cite{Legarra2014} but, to the best of our knowledge, 
has only been considered in plants in a study on wheat \cite{Ashraf2016} and
white spruce lines, respectively \cite{Ratcliffe2017}. 
Our objectives were threefold: (1) Examine whether the single step prediction
framework can be transfered to plant hybrids, (2) explore the utility of single 
step predition with a quantitative predictor and (3) evaluate the impact of the
genetic space covered by the incomplete predictor on predictive ability.










\section{Materials and Methods}
\Genetics2level{Maize hybrids}
The maize hybrid data set, subsequently denoted as \textit{Exp1}, comprised 
1,521 hybrids produced in 16 factorial mating designs between 142 Dent and 103 
Flint parent lines \cite{Westhues2017}. 
Best linear unbiased estimates (BLUEs) were computed across all factorials and 
three or more agro-ecologically diverse locations across Germany for six 
agronomically important traits by accounting for year, location, field 
replication, block and genotype effects as well as their interactions
(Table~\ref{table:TraitKey}).
All 245 parent lines have pedigree records back to the generation of their 
grandparents or deeper \cite{Westhues2017} and were genotyped using the Illumina 
BeadChip MaizeSNP50 \cite{Ganal2011}.
After filtering for minor allele frequency ($\geq 5$\%), heterozygosity rate
($\leq 5$\%) and call frequency ($\geq 95$\%), missing genotypes were imputed 
using the Beagle software \cite{Browning2009}.
The final number of polymorphic marker loci was 7,013 for the Dent and 6,212 for 
the Flint lines, respectively.
Gene expression data for seedlings seven days after sowing were obtained for a 
subset of 60 parental Dent and 43 parental Flint lines, that are parents of 685
hybrid progeny.
These data were obtained through two-color hybridizations using a custom 2K 
microarray (GPL22267) \cite{Westhues2017}.
After applying established normalization procedures \cite{Smyth2003,Ritchie2007},
gene expression BLUEs were computed for 1,323 transcripts while adjusting for 
putative effects arising from the presence of multiple factorials (\textit{cf.}
\citeNP{Westhues2017}) using the R-package \textit{limma} \cite{Ritchie2015a}.

This dataset was included to address objectives 1 and 2.



\Genetics2level{Maize diversity panel}
The maize inbred line data set, subsequently referred to as \textit{Exp2}, 
originally comprised 513 maize lines representing the global maize diversity 
and was reduced to the set of tropical and subtropical lines ($n = 211$) given 
that it is the largest of the four pre-classified subgroups \cite{Yang2014}.
We reduced the original dataset to this subset to avoid inflated predictive 
abilities due to large differences in the average performance of the different
subpopulations \cite{Windhausen2012}.
All inbred lines were evaluated in five Chinese environments ranging from
$18$ to $30^{\circ}$N and from $102$ to $110^{\circ}$E \cite{Yang2014}.
Best linear unbiased predictors (BLUPs) were calculated for 17 traits of which 
six are presented in this study (Table~\ref{table:TraitKey}).
The selection of these traits was done so that four traits were better 
predicted by transcriptomic data and two further traits were better predicted 
by genomic data.
MaizeSNP50 BeadChip-data were available for the entire set of lines from the
diversity panel.
Additionally, RNA sequencing was performed for 368 lines on young seedlings 15 
days after sowing.
By exploting identify by descent (IBD) between 49,728 SNPs on the BeadChip 
overlapping with the RNA-seq data, 556,809 high quality SNPs could be inferred 
for these lines \cite{Fu2013,Li2013}.
For the remaining 145 maize lines without RNA-seq data, high density markers
were inferred via projection of IBD regions onto the BeadChip, using this core 
set of markers \cite{Yang2014}.
The same SNP quality checks as for the parental inbred lines of the hybrids
were applied to the 211 inbred lines from the tropical/subtropical subset,
yielding 37,760 SNPs.
Gene expression data were normalized using a normal quantile transformation to
satisfy modelling assumptions \cite{Fu2013}.
A subset of 28,850 annotated genes was available for 149 out of the 211
genotypes and was kept for further analyses \cite{Li2013}.

This dataset was included to address objectives 2 and 3.

% Please add the following required packages to your document preamble:
% \usepackage{booktabs}
\begin{table}[]
\centering
\caption{Acronyms for the trait names}
\label{table:TraitKey}
\begin{tabular}{@{}lll@{}}
\toprule
Trait              & Acronym & Data Set  \\ \midrule
Cob weight         & CW      & Diversity \\
Days to silking    & DS      & Diversity \\
Ear diameter       & ED      & Diversity \\
100-grain weight   & GW      & Diversity \\
Kernel width       & KW      & Diversity \\
Plant height       & PH      & Diversity \\
Dry matter yield   & DMY     & Hybrid    \\
Dry matter content & DMC     & Hybrid    \\
Fat                & FAT     & Hybrid    \\
Protein            & PRO     & Hybrid    \\
Starch             & STA     & Hybrid    \\
Sugar              & SUG     & Hybrid    \\ \bottomrule
\end{tabular}
\end{table}

\Genetics2level{Population structure}
Separately for each data set, principal component analyses (PCA) were run on 
scaled and centered predictor matrices using the \emph{pca} function from the 
R-package \textit{LEA} \cite{Frichot2015}.
Least-squares estimates of ancestry proportions were estimated based on sparse 
nonnegative matrix factorization algorithms \cite{Frichot2014} using the 
\emph{snmf} function in \textit{LEA}.
Algorithms were run for $K=2, \dots 5$ putative ancestral gene pools with 25
repetitions for each value of $K$. 
For each $K$, only the repetition with the lowest cross-entropy was kept for 
further analyses.



\Genetics2level{Core sampling}
In order to evaluate the influence of the genetic constitution of the set of
genotypes that has only data on one out of two predictors, we generated core
samples based on two different approaches.
All core sets were assembled from the 149 maize inbred lines from the diversity
panel which had gene expression as well as genomic information.
We defined core samples as a subset of genotypes that is covered by two
predictors whereas all other genotypes (\textit{i.e.}, the complement) are only
covered by a single predictor.
The first application involved assembling core sets of varying sizes by 
maximizing the average genetic distance among core set members using the 
Modified Rogers (MR) distance as the criterion \cite{Thachuk2009}.
The size of the core set was varied in increments of ten percentage points and 
ranged from 10\% to 90\% of all 149 inbred lines.
For the second application, we first ran an admixture analysis in combination
with a PCA and kept $K=3$ as the number of putative ancestral population, which 
seemed to describe the population structure best.
Genotypes were then assigned to one of the three ancestral populations based on
whether any of the three relationship coefficients (A1-A3) was equal to or
greater than 0.5.
For example, a genotype with an admixture coefficient for A2 $\geq 0.5$ was
assigned to the second ancestral population.
Thereby, we obtained $n_{A1} = 17$, $n_{A2} = 109$, $n_{A3} = 19$ and four 
genotypes that could not be unambiguously assigned to any putative ancestral 
population based on this criterion.
To ensure commensurability of the three subpopulations within the set of 149
genotypes, the subset of 109 genotypes (\textit{i.e.}, members of A2) was
reduced to $n_{A2} = 19$ genotypes, again using the method of maximizing the
average genetic distance among selected genotypes described above.


\Genetics2level{Prediction}
\subsection{Kernels}
BLUP models are computationally attractive when the number of features $p$
exceeds the number of genotypes $n$ and are equivalent to a selection index
when fixed effects have been accomodated by the dependent variable 
\cite{Mrode2014}.
Depending on the data set, up to three predictors were available for agronomic
trait predictions, namely pedigree data (P), genomic data (G) and
transcriptompic data (T).
The corresponding feature matrix of the $l$-th group of inbred lines and the 
$o$-th predictor ($\mathbf{W}_{lo}$) has dimensions $n_{l} \times p_{o}$ where 
$n_{l}$ pertains to the number of genotypes in the $l$-th group and $p_{o}$
pertains to the number of features.
Note that $l$ is only defined for \textit{Exp1} where a distinction between 
the two heterotic groups is necessary whereas $l$ can be ignored for all 
models pertaining to the inbred lines of the maize diversity panel in 
\textit{Exp2}.
To avoid mentioning special cases, we will use the general notation with 
the index $l$ throughout this section, though.
For the diversity panel maize lines, $n_{l}$ corresponds to the number of 
genotypes whereas, in the case of the hybrid data, $n_{l}$ corresponds to the 
number of parent lines in the $l$-th heterotic group.
In generating kernels for each predictor, all features in $\mathbf{W}_{lo}$ were
centered and standardized to unit variance.
Then, each kernel can be defined as

\begin{equation} \label{eq:GenomicRelationship}
  \mathbf{K}_{lo} = \frac{1}{W_{lo}} \mathbf{W}_{lo} \mathbf{W}_{lo}^{\top},
\end{equation}

where $W_{lo}$ denotes the number of features \cite{VanRaden2008}.
In the case of pedigree data (P), coancestry coefficients were used directly
for $\mathbf{K}_{lo}$.





\subsection{Breeding value model}
The general model for breeding values was:

\begin{equation} \label{eq:KBLUPModel}
  \mathbf{y} = \mu + 
  \sum_{l=1}^{L} \mathbf{Z}_{lo} \mathbf{g}_{lo} +
  \mathbf{\epsilon},
\end{equation}


where $\mathbf{y}$ is the vector of observed inbred line or hybrid performance,
respectively, $\mu$ is the fixed model intercept, and $\epsilon$ is the residual
error.
For the hybrid data from \textit{Exp1}, we fit separate additive effects for 
each of the $L$ heterotic groups and $\mathbf{g}_{lo}$ now pertains to the 
general combining ability effects of the parental inbred lines (either Dent or 
Flint, as expressed by $l$).
For the maize diversity panel inbred lines from \textit{Exp2}, we fit random 
genotype effects $\mathbf{g}_{lo}$ (\textit{i.e.}, breeding values) of the 
lines for the $o$-th predictor.
For both experiments, genetic effects $\mathbf_{g}_{lo}$ are associated with 
the vector of phenotypic performance $\mathbf{y}$ through a corresponding 
design matrix $\mathbf{Z}_{lo}$.
The random effects ($\mathbf{g}_{lo}$) have expectation zero and variance equal 
to $\mathbf{K}_{lo} \sigma^{2}_{{g}^{lo}}$ and $\mathbf{I} \sigma^2_{\epsilon}$ 
for the residual error.
All prediction models were implemented using the R-packages \textit{BGLR}
\cite{Perez2014} and \textit{sspredr}
\url{https://github.com/mwesthues/sspredr}. 





\subsection{Single step prediction}
The breeding values in a prediction model are in general given by:
\begin{equation} \label{eq:mrnaebv}
	\mathbf{\hat{g}} = \mathbf{W}\boldsymbol{\hat{\alpha}},
\end{equation}

where $\boldsymbol{\hat{\alpha}}$ is the solution to a ridge regression model,
which is equivalent to the BLUP model.

Consider now the situation in which one predictor is complete in the sense that
it contains information on the whole population whereas a second predictor is
incomplete and covers only a subset of the population.
We can take a similar route as in \citeNP{Fernando2014} and impute covariates 
of the incomplete predictor by using covariates of the complete predictor.
Let the subscript $1$ denote individuals covered only by the complete
predictor whereas individuals covered by both, the complete and the
incomplete predictor, are indicated by the subscript $2$.
Further, the covariates in $\mathbf{W}$ are centered.
The vector $\mathbf{g}_1$ can be written as the sum of the conditional 
expectation given $\mathbf{g}_2$ and a residual error term:
\begin{align} \label{eq:mrna1}
	\mathbf{g_1} &= E(\mathbf{g}_1|\mathbf{g}_2) + \boldsymbol{\epsilon} \\
	&= \mathbf{K_{12}}\mathbf{K_{22}}^{-1}\mathbf{W_2}\boldsymbol{\hat{\alpha}} + (\mathbf{g_1} - \mathbf{K_{12}}\mathbf{K_{22}}^{-1}\mathbf{W_2}\boldsymbol{\hat{\alpha}}) \\
	&= \mathbf{\hat{g}}_1 + \boldsymbol{\epsilon}
\end{align}

The covariance matrix of $\boldsymbol{\epsilon}$ is $\mathbf{K}_{11} - \mathbf{K}_{12}\mathbf{K}_{22}^{-1}\mathbf{K}_{21} = (\mathbf{K}^{11})^{-1}$
\cite{Legarra2009}.

The structure of the residual imputation/prediction error is known and can 
therefore be modelled.
The covariates not covered by the incomplete predictor can be predicted using 
the expectation of a multivariate normal random vector given correlated 
observations, which are related to the Best Linear Predictor.

The general model for the single step procedure can then be written as 

\begin{equation} \label{eq:single-step-model}
\mathbf{y} = \mathbf{Xb} + \mathbf{W} \boldsymbol{\alpha} + \mathbf{U} \boldsymbol{\epsilon} + \mathbf{e},
\end{equation}

with

\begin{equation} \label{eq:single-step-submatrices}
\mathbf{X} = 
\begin{bmatrix}
  X_1 \\
  X_2 
 \end{bmatrix},
 \mathbf{W} = 
\begin{bmatrix}
  Z_1\hat{\mathbf{W}_1} \\
  \mathbf{W}_2 
 \end{bmatrix},
 \mathbf{U} = 
\begin{bmatrix}
  Z_1 \\
  0 
 \end{bmatrix}
\end{equation}

Following the notation in \citeNP{Fernando2014} the phenotypes can be untangled 
like this:

\begin{align} \label{eq:entangled-augmented-single-step-model}
\begin{bmatrix}
  y_1 \\
  y_2 
 \end{bmatrix}
& =
 \begin{bmatrix}
  X_1 \\
  X_2 
 \end{bmatrix}
 \boldsymbol{\beta} + 
 \begin{bmatrix}
  Z_1 & 0 \\
  0 & Z_2 
 \end{bmatrix}
\begin{bmatrix}
  g_1 \\
  g_2 
 \end{bmatrix}
  + \mathbf{e} \\
    & = 
 \begin{bmatrix}
  X_1 \\
  X_2 
 \end{bmatrix}
 \boldsymbol{\beta} + 
 \begin{bmatrix}
  Z_1 & 0 \\
  0 & Z_2 
 \end{bmatrix}
\begin{bmatrix}
  \mathbf{K}_{12}\mathbf{K}_{22}^{-1}\mathbf{W}_2\boldsymbol{\alpha} + \boldsymbol{\epsilon}  \\
  \mathbf{W}_2\boldsymbol{\alpha} \\
 \end{bmatrix}
  + \mathbf{e} \\
    & = 
 \begin{bmatrix}
  X_1 \\
  X_2 
 \end{bmatrix}
 \boldsymbol{\beta} + 
 \begin{bmatrix}
  Z_1 & 0 \\
  0 & Z_2 
 \end{bmatrix}
\begin{bmatrix}
  \hat{\mathbf{W}_1}\boldsymbol{\alpha} + \boldsymbol{\epsilon} \\
  \mathbf{W}_2\boldsymbol{\alpha} \\
 \end{bmatrix}
  + \mathbf{e} \\
\end{align}


Our final model looks like this:

\begin{align} \label{eq:final-model}
\mathbf{y} &= \mathbf{Xb} +
\sum_{l=1}^{L} \mathbf{Z}_{l}\mathbf{W}_{l} \boldsymbol{\alpha}_{l} + 
\sum_{l=1}^{L} \mathbf{Z}_{l}\mathbf{U}_{l} \boldsymbol{\epsilon}_{l} +
\mathbf{e}.
\end{align}


Breeding values can be estimated as:

\begin{equation} \label{eq:breeding-values}
\hat{\mathbf{g}} = 
 \begin{bmatrix}
  \hat{\mathbf{W}_1} \\
  \mathbf{W}_2 
 \end{bmatrix}
 \hat{\boldsymbol{\alpha}}
 + 
 \begin{bmatrix}
  \mathbf{Z}_1 \\
  0
 \end{bmatrix}
 \hat{\boldsymbol{\epsilon}} 
\end{equation}

For the hybrid data, the obtained GCA effects for the inbred lines represent 
half their breeding values.
The predicted agronomic performance is:

\begin{equation} \label{eq:predicted-performance}
\hat{\mathbf{y}} = \sum_{l=1}^{L} \mathbf{Z}_{l}\hat{\mathbf{g}}_{l} 
\end{equation}




\subsection{Predictive ability and model validation}
For the validation of our predictions, we employed leave-one-out
cross-validation (LOOCV) routines.
In the case of the diversity panel maize inbred lines, LOOCV was performed by 
using a single genotype as a hold-out sample, which will subsequently be 
predicted, while using all other inbred lines for model training.
This process is repeated until all inbred lines have been used once for testing
and $n - 1$ times for model training.

For the hybrid data, LOOCV was carried out as follows:
Let $D$ and $F$ denote the set of parental inbred lines from the Dent and the 
Flint group, respectively.
Further, let $H \cap [D \times F]$ denote the set of hybrids from crosses 
between $D$ and $F$.
The training set ($H_{TRN}$) for the hybrid to be predicted ($H_{ij}$) --- 
where $i \in D_{TRN}$ and $j \in F_{TRN}$ --- was assembled as 
$H_{TRN} = [H \cap (D_{TRN}^{C} \times F_{TRN}^{C})]$ with
$D_{TRN}^{C} = D \setminus D_{TRN}$ and $F_{TRN}^{C} = F \setminus F_{TRN}$.

We judged the performance of each model by looking at its predictive ability,
which is calculated as $\rho(\mathbf{y}, \mathbf{\hat{y}})$, where 
$\mathbf{\hat{y}}$ is the vector of predicted values from each LOOCV run.
Our confidence in the performance of each model was evalutaed by computing the
coefficient of variation for the predicted values:

\begin{equation} \label{eq:CV}
CV = \sqrt{\frac{\sum_{i = 1}^{n} \sigma^{2}_{\hat{y}(i)} / n}{\sum_{i=1}^{n}\hat{y}_{i} / n}},
\end{equation}

where $n$ denotes the number of genotypes (\textit{i.e.}, inbred lines or 
hybrids).












\section*{Results}
\Genetics2level{Inbred lines}
At the level of the 149 genotypes for which both, genomic (G) and 
transcriptomic (T) information, were available, the use of T yielded higher
predictive abilities than the use of G for four out of six traits
(Fig. \ref{fig:InbredResults}A).
In terms of predictive ability, the absolute advantage of T over G was as large 
as 0.04 as observed for the traits DS, ED, GW and KW.
Except for KW, the predictive ability for predictions based on G were always
higher for the full set of inbred lines ($n = 211$) than for the reduced set of 
inbred lines ($n = 149$).
The combination of G and T (\textit{i.e.}, GT) in a single-step approach for the 
prediction of all 211 lines, performed at least as well as G alone when T was 
superior over G for the reduced set of 149 lines.
In the case of GW and ED, the use of GT yielded a slight improvement in predictive 
ability over G.
Where T was a worse predictor for the reduced set of 149 lines than G,
predictive abilities of GT for the full set of 211 lines were also worse than 
those based on G alone.

% Script: ./analysis/maizego_core_fraction_plot_predictive_abilities.R
\begin{figure}[H]
  \centering
  \includegraphics{./tables_figures/inbred_line_combi_plot.pdf}
  \caption{
  Predictive abilities and average coefficient of variation for
  tropical/subtropical inbred lines from the maize diversity panel and six 
  agronomic traits.
  As predictors, genomic (G) data, transcriptomic (T) data and their 
  combination (GT) were used.
  (\textbf{A}) The 'Core' panel includes a subset of 149 lines covered by
  genomic and transcriptomic information whereas the 'Full' panel comprises all
  211 genotypes. In the 'Full' set of genotypes, transcriptomic records were
  imputed for 62 lines.
  (\textbf{B}) Influence of structural missingness in the coverage of genotypes 
  with transcriptomic information on single step-based predictive abilities 
  using a 'Core' set of 149 inbred lines.
  Transcriptomic records were deleted for 17 to 19 genotypes that were
  primarily associated with a particular subgroup in the material, as defined
  by an admixture analysis and a PCA (Fig. \ref{fig:PopStructure}).
  (\textbf{C}) Influence of core set sampling on predictive abilities using
  nine different sub set sizes where 'Core Fraction' indicates the fraction
  of 149 inbred lines that is covered by both, genomic and transcriptomic
  data.
  } 
  \label{fig:InbredResults}
\end{figure}



\Genetics2level{Hybrids}
The maize hybrid data set contains information on six agronomic traits and
three predictors, namely pedigree data (P), genomic information (G), 
transcriptomic information (T), and combinations thereof.
Pedigree records and genomic information were available for the 245 parent 
lines of all 1,521 hybrids and we denote these data as the 'Full' set whereas
transcriptomic data were available for 103 parent lines of 685 hybrids, which
we denote as 'Core'.
For DMY, T was the best predictor in the core set of 685 hybrids with a 
predictive ability that was 0.08 points higher than for genomic data, as the 
second best predictor for this trait (Fig. \ref{fig:HybridResults}).
In the core set, P had the highest predictive ability for STA and DMC.
For the other three traits, G was the superior predictor inside the core set.
In the full set of 1,521 hybrids, predictive abilities obtained with G rose 
markedly for all traits with the exception of FAT.
The combination of genomic and transriptomic information (GT) in a single-step
model yielded a small improvement over G alone for the traits DMY and PRO.
Except for PRO, compared to using only G, trait prediction did not benefit from 
the single step approach when T was not superior to G in the core set.
Predictive abilities resulting from combinations of pedigree with genomic (PG)
information were better than P alone for all traits and PT improved upon P in
the core set for four traits.

% Script: ./analysis/uhoh_plot_predictive_abilities.R
\begin{figure}[H]
\centering
  \includegraphics{./tables_figures/hybrid_predictive_ability.pdf}
  \caption{
    Predictive abilities and average coefficients of variation for the set of 
    maize hybrids and six agronomic traits.
    As predictors, pedigree (P), genomic (G), transcriptomic (T) data and 
    combinations thereof we used.
    The 'Full' set of genotypes comprises 1,521 hybrid progeny based on 245
    parental inbred lines, whereas the 'Core' set comprises 685 hybrid progeny
    based on 103 parental inbred lines.
  } 
  \label{fig:HybridResults}
\end{figure}





\Genetics2level{Coverage of the genetic space}
Single-step approaches are based on the use of one complete predictor that
covers all genotypes and one incomplete predictor that covers only a subset of
all available genotypes.
Here, we explore the impact of which data are covered by the incomplete 
predictor on the predictive abilities obtained via single-step approaches by
focusing exclusively on the 149 inbred lines from the maize diversity panel
that are covered by both, genomic and transcriptomic information.
First, we assembled nine different core sets, which we defined as subsets
of genotypes that maximize the average genetic distance between genotypes of
the full set.
The nine core sets differ in that each covers a different number of genotypes,
ranging from only 10\% to 90\% of all genotypes covered by the incomplete
predictor (Fig. \ref{fig:CoreSetPCA}).



\begin{figure}[H]
\centering
\includegraphics{./tables_figures/maizego_core_sampling_pca.pdf}
\caption{
  Coverage of the genetic space by nine different core sets. 
  Genotypes are distributed according to their scores based on the first two
  principal components (PC).
  The total genetic space was spanned by 149 genotypes from the maize diversity 
  panel for which both, genomic and transcriptomic data, were available.
  Genotypes that were members of the core set (orange triangles) were declared 
  as being covered by both predictors whereas members of the complement (blue,
  filled circles) were declared as only being covered by genomic information.
  Assignments of genotypes to the core set was done by selecting a subset of
  genotypes that maximized the average genetic distance in the core set based 
  on the Modified Rogers distance between all genotypes.
  Core set sizes ranged from 10\% to 90\% of the 149 available genotypes and
  are given as decimals above each facet of the plot.
}
\label{fig:CoreSetPCA}
\end{figure}


Predictive abilities obtained with the different core sets varied widely within
traits (Fig. \ref{fig:InbredResults}B).
For the traits CW and PH, which were better predicted via G than via T in the 
reduced set of lines, predictive abilities were higher when only a small number 
of genotypes was included in the core set.
This corresponds to a situation where a large number of genotypes is being 
imputed and the majority of the information is derived from genomic information. 
For most other traits, an initial increase in predictive abilities when moving
from a core set size of 0.1 to 0.5/0.6 was observed, followed by a drop around
0.6/0.7 and a subsequent hike up to a core set size of 0.9.
A principal component analysis depicts the coverage of the genetic space by
genotypes (Fig. \ref{fig:PopStructure}).
With a core set size of 50-60\% of the 149 genotypes, the congruence between
core set and complement was high whereas genotypes in the south-west direction
of the PCA plot were overrepresented compared to genotypes in the east 
direction at a core set size of 0.7.

As an alternative to exploring the impact of genotypes sampled based on the
maximization of genetic distance within the core set, we wanted to explore a
more extreme scenario where we select core sets based on population structure.
Hereto, we ran an admixture analysis on the same set of 149 inbred lines 
assuming $K=3$ ancestral populations (Fig. \ref{fig:PopStructure}).

\begin{figure}[H]
\centering
\includegraphics{./tables_figures/maizego_admixture_k3_pca.pdf}
\caption{
  Population structure of 149 maize inbred lines from the maize diversity panel
  covered by genomic and transcriptomic data.
  (\textbf{A}) Ancestry coefficients from an admixture analysis when 
  considering $K=3$ ancestral populations.
  (\textbf{B}) Principal component analysis with genotypes colored according to 
  the most likely ancestral population (\textit{i.e.}, share $\geq 0.5$) as 
  indicated by the admixture analysis.
  Genotypes for which none of the three ancestry coefficients reached a value
  of 0.5 were classified as "mixed".
}
\label{fig:PopStructure}
\end{figure}


Of the 149 genotypes, 17 belonged to more than 50\% to the first ancestral
population (A1), 109 belonged to more than 50\% to the second ancestral
population (A2) and 19 belonged to more than 50\% to the third ancestral
population while four genotypes could not be unequivocally assigned to either
ancestral population (mixed).
To reconcile the large size of the A2 set compared to A1 and A3, we selected 19 
genotypes from A2 using previously described core sampling.
For single-step prediction, each of the three genotype sets were then declared
as having genomic and transcriptomic information whereas their complements were
covered by transcriptomic data only.
Genotypes from A1 and A3, which were at the far ends in the PCA plot (Fig.
\ref{fig:PopStructure}B), were estimated with less precision than genotypes
from the central cluster of genotypes belonging largely to the A2 population
(Fig. \ref{fig:PopStructure}A, Fig. \ref{fig:InbredResults}C).

Regarding the hybrid material, a PCA carried out separately for the two groups
of parental inbred lines showed that the entire genetic space in each group 
was covered well by genotypes that had both, transcriptomic and genomic data 
(Fig. \ref{fig:PCA}).

\begin{figure}[H]
\centering
\includegraphics{./tables_figures/diversity_dent_flint_pca.pdf}
\caption{
  Principal component (PC) analysis i) of the 211 tropical/subtropical lines 
  from the maize diversity panel, ii) of the 142 Dent parent lines of the 1,521
  maize hybrids and iii) of the 103 Flint parent lines of the maize hybrids.
  All inbred lines, which are covered by genomic and transcriptomic features
  are depicted as orange triangles whereas genotypes for which only genomic
  information are available are depicted as blue dots.
}
\label{fig:PCA}
\end{figure}













\section*{Discussion}
\Genetics2level{Advantages of marker effect models}
Single step approaches in hybrid populations have so far been limited to pig
breeding \cite{Xiang2015,Xiang2016,Tusell2016}.
All of these studies used breeding value-based single step BLUP methods
(SS-BLUP) conceived by \citeNP{Legarra2009} and \citeNP{Christensen2010} with
extensions to hybrid populations \cite{Christensen2014,Christensen2015}.
An attactive property of SS-BLUP models is their computational advantage over 
marker-effect single step Bayesian regression models (SSBR) when the number of 
genotypes is smaller than the number of features of the incomplete predictor.
Nevertheless, SS-BLUP suffers from a variety of issues, which we will present
briefly:
First, the combination of the numerator with the genomic relationship matrix
requires commensurability of the two matrices, which necessitates some form of
scaling and weighting of $\mathbf{A}$ and $\mathbf{G}$
\cite{Christensen2012,Christensen2012a}.
Second, the fact that mostly recent individuals have genotypes whereas the 
majority of old individuals has only pedigree records introduces a bias that 
has to be accounted for \cite{Vitezica2011,Legarra2015,Garcia-Baccino2017}.
Third, the addition of new genotypes requires updates of $\mathbf{G}$ that
demand additional measures such as computing the inverse of $\mathbf{G}$ via
recursion
\cite{Misztal2014,Misztal2016,Misztal2016a,Fragomeni2015,Masuda2016,Pocrnic2016}.
Advantages of the SSBR algorithm, conceived by \citeNP{Fernando2014}, include
that i) it does not require commensurability of $\mathbf{A}$ and $\mathbf{G}$, 
ii) it models the bias incurred by the incomplete predictor explicitly as 
$\mathbf{\epsilon}$ and iii) its speed depends largely on the number of 
features.
To our best knowledge, this study presents the first implementation of the SSBR 
algorithm on a hybrid data set.




\Genetics2level{Addition of new predictors}
Pedigree records are the most ubiquitous predictor in animal and plant breeding
programs, because they are easy and inexpensive to collect.
Compared to pedigree information, which represent the expected relationship
among individuals, genomic information capture Mendelian sampling and thereby 
provide an improved proxy of the realized relationship among individuals.
Nevertheless, genomic data do not exhaustively capture physiological epistasis 
\cite{Jiang2015,Guo2016,Vazquez2016}, describing the interactions 
within and between different biological strata \cite{Sackton2016}.
Such interactions were found to be pervasive throughout the genomes of yeast 
\cite{Brem2005} and humans \cite{Brown2014} and have motivated studies on the 
utility of downstream 'omics' predictors for integrating such interactions.
Recently, encouraging results have been found for humans \cite{Vazquez2016}, 
maize inbred lines \cite{Guo2016} and maize hybrids \cite{Westhues2017}.
While single step prediction has become an established method when complete
pedigree but only incomplete genomic records are available, its application has
not yet been considered for the imputation of a complete predictor based on 
incomplete downstream 'omics' predictors such as transcriptomic, proteomic or 
metabolomic data.
Here, we considered single step approaches for a maize inbred line diversity
panel as well as for a collection of maize hybrids using pedigree, genomic and
transcriptomic data as predictors. 


\Genetics2level{Superiority of single step models}
\subsection{Inbred lines}
When using a single step approach based on complete genomic and incomplete
transcriptomic information, gains in predictive ability for the inbred lines
were at best just slightly higher than when using complete genomic information
alone.
One of the reasons might be that the subset of genotypes that was covered by
transcriptomic data was not a good representation of the full genetic space of
all available genotypes.
Another possibility is that the number of additional phenotypes in the full
set compared to the core set ($\Delta(n) = 62$) was not large enough to provide
considerably more useful information.
The latter was the case for \citeNP{Ashraf2016}, who imputed genotypes for about 
10,000 wheat lines using pedigree information and observed greater accuracy of
the single step method over genomic BLUP for all four evaluated traits.
Predictive abilities reported for the maize diversity panel in this study were
slightly different from those reported by \citeNP{Guo2016}, for three
possible reasons:
1) We reduced the dataset to the 211 tropical/subtropical lines and excluded 
all other genotypes from our analyses whereas \citeNP{Guo2016} used genotypes 
from four different subpopulations, modeled as a fixed effect, and used only a 
subset of 157 genotypes from the set of tropical/subtropical lines.
2) We applied quality checks for the predictor data after generating the subset
of 211 inbred lines, whereas \citeNP{Guo2016} applied the quality checks to 368
genotypes.
3) While \citeNP{Guo2016} used five-fold cross-validation with 500 repetitions,
we employed a leave-one-out cross-validation scheme (LOOCV).
Notwithstanding, the relative differences between predictive abilities for G 
and T, respectively, were similar in both studies.


\subsection{Maize hybrids}
Hybrid breeding --- with noteworthy commercial applications in pig
\cite{Xiang2016,Tusell2016} and maize breeding \cite{TheRoyalSociety2009} --- 
is a particularly challenging application field for prediction tools 
\cite{Kadam2016}.
Here, $2n$ parent individuals from two genetically distinct heterotic groups are
crossed to each other, yielding $n^{2}$ potential hybrid progeny that would
require intensive field testing.
In medium-sized plant breeding programs, the advent of the doubled haploid (DH) 
technology \cite{Wedzony2009} now allows for an annual production of 1,000 
parent lines in each heterotic group, amounting to $10^{6}$ putative hybrid
progeny.
\citeNP{Westhues2017} have shown that, in hybrid breeding, the probability 
of successfully selecting the best observed genotypes based on the best 
predicted candidates is a strongly convex function of the predictive ability.
Thus, even the minor gains when using transcriptomic and genomic data in a
single step might justify additional investments in RNA-seq at least of a
subset of genotypes that represents the genetic space of the breeding 
material well.
For the hybrid data, the predictive abilities of genomic information
for the whole set of available parent lines rose considerably for several
agronomic traits.
Hence, the inclusion of trancsriptomic data could not further improve upon
this result.
Predictive abilities obtained when using pedigree data could be 
improved considerably when, depending on the agronomic trait, combined with 
either transcriptomic or genomic information.
Evaluations of core sets for the parental inbred lines of the maize hybrids were 
outside the scope of this study, but results on the inbred lines have shown that 
homogeneous coverage of the genetic space by the incomplete predictor can 
considerably leverage the predictive ability in single step prediction compared 
to using only the complete predictor.


\Genetics2level{Coverage of the genetic space}
Whereas animal breeding populations are oftentimes so large that the assembly 
of a core set of individuals with records on multiple predictors can effectively 
be done at random once more than 10,000 animals have pedigree records 
\cite{Fragomeni2015,Lourenco2015,Masuda2016}, pedigrees in plant breeding 
programs are typically considerably smaller.
Single step prediction offers the promise of leveraging the predictive ability
for all genotypes by borrowing the superior performance of an auxiliary
predictor on only a subset of genotypes, thereby reducing predictor costs.
Hence, we were interested in determining which individuals should be 
complemented with information on another predictor.
When using core sets of varying sizes, which were established with the
constraint of maximizing the average genetic distance among their members, we
noticed strong fluctuations in predictive abilities when moving from small core
sets to large core sets.
For traits that were better predicted by transcriptomic than by genomic
information in the core set, predictive abilities typically peaked around
core set size of 0.5/0.6, then dropped sharply at 0.7 and started to increase
again with a size of 0.9.
A principal component analysis showed that, at a core set size of 0.7, 
genotypes assigned to the A3 ancestral population (Fig. \ref{fig:PopStructure})
were largely elements of the core set whereas genotypes with a large share at 
the A1 ancestral population were predominantly excluded from the core set.
At a core set size of 90\% of the original size, the majority of the A1
genotypes were finally covered by the core set as well.
This result suggests a negative bias in the imputation of genotypes from the A1
population and could explain the drop in predictive ability at a core set size
of 0.7, which is in agreement with the reduced predictive abilities observed
when stripping A1 genotypes from their transcriptomic information (Fig.
\ref{fig:InbredResults}B).



\section*{Conclusions}
We have successfully applied single step prediction to a hybrid plant data set 
while imputing with a quantitative downstream 'omics' predictor.
By declaring different subsets of individuals as covered by one or two 
predictors, respectively, we could show the importance of homogeneous coverage
of the genetic space by genotypes covered by both predictors on predictive
ability.
Extensions to more than two predictors were outside the scope of this study
but, with mounting interest in systems genetics, should be considered in the
future.





\section{Acknowledgments} 
This project was funded by the German Federal Ministry of Education and 
Research (BMBF) within the projects OPTIMAL (FKZ: 0315958B,0315958F),
SYNBREED (FKZ: 0315528D) and by the German Research Foundation 
(DFG, Grants No. ME 2260/5-1 and SCHO 764/6-1).
The authors acknowledge support by the state of Baden-W{\"u}rttemberg through 
bwHPC.
Financial support for M.W. was provided by the Fiat Panis foundation, Ulm,
Germany.



\nolinenumbers
% Bibliography
\bibliography{library}
\bibliographystyle{mychicago}
\end{document}
